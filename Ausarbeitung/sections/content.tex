%% LaTeX2e class for seminar theses
%% sections/content.tex
%% 
%% Karlsruhe Institute of Technology
%% Institute for Program Structures and Data Organization
%% Chair for Software Design and Quality (SDQ)
%%
%% Dr.-Ing. Erik Burger
%% burger@kit.edu
%%
%% Version 1.0, 2018-04-16

\section{Bitcoin}
\label{ch:FirstContentSection}

%% -------------------
%% | Example content |
%% -------------------
The content chapters of your thesis should of course be renamed. How many
chapters you need to write depends on your thesis and cannot be said in general.

Of course, you can split this .tex file into several files if you prefer. 


\subsection{Merkle Tree}
\label{sec:FirstContentSection:FirstSubSection}

\dots

\subsection{Probleme}
\label{sec:FirstContentSection:FirstSubSection}

Feste Blockgröße -> Beschränkte Transaktionsgeschwindigkeit, Malleability, \dots

\section{SegWit}
\label{ch:SecondContentSection}

\subsection{Funktionsweise}
\label{sec:SecondContentSection:FirstSubsection}

\dots

\subsubsection{Backwards Compatibility}

\subsection{Implementierung?}
\label{sec:SecondContentSection:SecondSubsection}

\dots

\subsection{Geschichte}

\dots

\subsubsection{Unvollständige Inbetriebnahme}

Add additional content sections if required by adding new .tex files in the
\code{sections/} directory and adding an appropriate 
\code{\textbackslash input} statement in \code{thesis.tex}. 
%% ---------------------
%% | / Example content |
%% ---------------------