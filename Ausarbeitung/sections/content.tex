%% LaTeX2e class for seminar theses
%% sections/content.tex
%% 
%% Karlsruhe Institute of Technology
%% Institute for Program Structures and Data Organization
%% Chair for Software Design and Quality (SDQ)
%%
%% Dr.-Ing. Erik Burger
%% burger@kit.edu
%%
%% Version 1.0, 2018-04-16

\section{Transaction Malleability}
\label{ch:TransactionMalleability}

\todo{Hierhin noch Zusammenfassung des Kapitels}

\subsection{Transaction ID}
\label{sec:TransactionMalleability:TransactionID}
Die Transaction ID ist als eindeutiger Identifikator für eine Transaktion gedacht. Diese wird generiert indem die Transaktion, bestehend aus unter anderem den Inputs, Outputs doppelt mit SHA-256 gehasht wird. Die Transaktionsdaten enthalten auch ein Skript, welches zur Verifizierung der Transaktion verwendet wird. Hierfür wird häufig ScriptSig verwendet, welches mithilfe von einem privaten Schlüssel die Nachricht signieren lässt und mithilfe eines öffentlichen Schlüssels die Signatur verifizieren lässt.

\subsection{Problem}
\label{sec:TransactionMalleability:Problem}
Die Transaction Malleability ist ein Problem des Bitcoin, bei dem die Transaction ID von einer Person geändert werden kann, die nicht der Sender einer Transaktion ist. Hierfür wird genutzt, dass das Skript zur Verifizierung der Transaktion mitgehasht wird und somit direkten Einfluss auf die Transaction ID hat. Dieses Skript kann nun zum Beispiel abgeändert werden, indem eine einzige Leerzeile zum Skript hinzugefügt wird (Diese verändert nicht die Funktion des Skripts, sondern nur das Aussehen). Da bei der Hash-Funktion SHA-256 jedoch jede noch so kleine Änderung der Eingabe-Daten zu einem völlig anderen Hash als Ergebnis führt wird hierdurch die Transaction ID völlig verändert.

\subsection{Angriffe und MtGox}
\label{sec:FirstContentSection:Angriffe}
An einem Beispiel verdeutlicht sich am besten, wie der Angriff funktioniert:
Alice sendet Bob einen Bitcoin mittels einer Transaktion, die mit dem Bitcoin-Netzwerk geteilt wird.
Daraufhin verfälscht Bob, der hier den Angreifer spielt, die Transaction ID unter Ausnutzung der Transaction Malleability und teilt seine selbst erstellte Transaction mit dem Netzwerk. In dieser Transaktion erhält Bob immernoch einen Bitcoin und zwar von dem selben UTXO, den Alice in ihrer Transaktion verwendet hat.
Falls nun Bob's Transaktion akzeptiert wird, bevor die von Alice veröffentlichte Transaktion in einem neuen Block geteilt wird, dann wird Alice's Transaktion von allen Minern abgelehnt, da der verwendete UTXO in Bob's Transaktion bereits verwendet wurde.
Darauf folgend teilt Bob Alice mit, dass die Transaktion vom Netz abgelehnt wurde, woraufhin Alice die Transaktion mit ihrer Transaction ID sucht und stellt fest, dass diese tatsächlich abgelehnt wurde. Nun sendet Alice erneut einen Bitcoin an Bob, diese Transaktion wird akzeptiert und Bob hat somit 2 Bitcoin statt einem erhalten, ohne dass Alice dies bemerkt hat. \\
Man sieht, dass dies kein technischer, sondern ein sozialer Angriff ist, da Bob Alice dazu bringt, ihm einen 2. Bitcoin zu senden und ihn nicht direkt von ihr stiehlt. \\
Ein bekannter Fall ist MtGox, die im Jahr 2014 der Presse erklärten, dass sie vor allem aufgrund von häufigen Angriffen, die durch Transaction Malleability möglich gemacht wurden, Bankrott gegangen sind. \cite{springer:malleability_and_mtgox}
Noch im selben Monat des angemeldeten Bankrotts hatte MtGox fast 70\% aller bisherigen Handel mit Bitcoin abgeschlossen. \cite{springer:malleability_and_mtgox}


\section{Segregated Witness}
\label{ch:SegWit}

\dots

\subsection{Idee}
\label{sec:SegWit:Idee}

\dots

\subsubsection{Backwards Compatibility}
\label{sec:SegWit:Idee:BackwardsCompatibility}

\dots

\subsection{Implementierung}
\label{sec:SegWit:Implementierung}

\dots

\subsection{Folgen für Transaction Malleability}
\label{sec:SegWit:Folgen}

\dots

\subsection{Geschichte}
\label{sec:SegWit:Geschichte}

\dots

%\subsubsection{Unvollständige Inbetriebnahme}